\pagestyle{plain}
\chapter*{Résumé}
Le suivi de seiche dans des vidéos sous-marines prises en conditions réelles est difficile. Ceci est dû aux mouvements de la caméra, au contraste, ou encore à la colorimétrie, qui peut varier d'une vidéo à l'autre.\\
Notre objectif est donc de proposer une solution robuste pour le suivi de seiche en environnement non contrôlé.\\
\\
Nous avons commencé par étudier la littérature dans le domaine du suivi d'objet, afin d'avoir une idée des différentes techniques, ainsi que de leur efficacité.\\
\\
Nous utilisons un filtre à particule comme base de notre algorithme de suivi, il utilisera plusieurs types de descripteurs qui seront comparés entre eux afin de déterminer les plus efficaces. Leur efficacité est mesurée en utilisant la méthode de Pascal VOC, qui consiste à comparer la bounding box de référence avec celle estimée par notre méthode.\\
\\
Le suivi est amélioré grâce à une étape de prédiction qui permet de gérer les cas d'occlusions et de déformation des seiches au cours de la vidéo.\\
\\
Le modèle d'intelligence artificielle YOLOv7 est utilisé pour faire la détection initiale de seiches dans les premières frames de la vidéo de manière automatique.\\
Ce modèle sera également utilisé en tant qu'algorithme de suivi à part entière, et des résultats préliminaires ont été établis.\\


\clearpage
